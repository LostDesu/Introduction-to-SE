\documentclass[a4paper,12pt]{article} 
\usepackage[T2A]{fontenc}			
\usepackage[utf8]{inputenc}			
\usepackage[english,russian]{babel}	
\usepackage{amsmath,amsfonts,amssymb,amsthm,mathtools} 
\usepackage{wasysym}
\usepackage{amsmath}

\author{конспект от TheLostDesu}
\title{Введение в программную инженерию}
\date{\today}


\begin{document}
\maketitle
\section{Рождение программирования}
\subsection{Программа}
В 1633г. в Англии впервые использовано слово $program$. Это было устное объявление властей, которое использовалось до печати распоряжений. \\
В словаре Даля было слово Програма\footnote{не опечатка} - Краткий очерк, начертание, содержание сочинения; План торжества, зрелища; задача, пояснительная записка на работу\\
\subsection{Автомат}
Автомат - нечто, запущенное единожды не требует вмешательства человека для работы.
Первые автоматы появились еще давно. Часы, например - автомат.
Около 100г. Герон Александрийский(Механик), написал сочинение Автоматы. Также он впервые использовал накопление энергии, возникающее при подъеме груза. Также он придумал использовать неоднородную намотку веревки на барабан, что позволяло грузу падать с разным ускорением. Также возникла идея универсализаии, ведь веревку можно было наматывать по разному, что давало разные конфигурации вращения.\\
В средние века получают популярность двигающиеся фигурки и музыкальные автоматы. 
В 18м веке начался рассвет автоматов. Появились очень сложные фигуры. Например барабанщик, бьющий в барабан, флейтист, играющий на флейте, утка клюющая еду. Появился даже автомат - рисовальщик, создающий рисунки. Появились автоматические ткацкие станки. Он использовал перфоленту для плетения нитей. После перфоленты заменили на набор перфокарт. Это еще больше универсализировало устройства.
\subsection{Арифметическая машина}
Около 1640г Блез Паскаль в 17м веке разработал принципы арифметических машин. При помощи инженеров было создано 50 "Паскалин" - арифметических машин. Однако, рыночного успеха не обрела. В 170х годах изобретается первый арифмометр - он мог не только складывать, но и умножать и делить. В 1786г. была изобретена "Разностная машина" - машина умеющая вычислять функции. 
\end{document}